% ==========================================
% CONFIGURACIÓN DEL PROYECTO
% ==========================================
%
% Este es el ÚNICO archivo que necesitas editar para
% personalizar la plantilla a tu proyecto.
%
% TIP: No modifiques archivos en common/ ni config/configuracion.tex
%      a menos que necesites funcionalidad avanzada.
%
% TIP: Los campos marcados con (*) son obligatorios.
%      Los demás puedes dejarlos vacíos con {}
% ==========================================

% --- Información del documento (*) ---
\newcommand{\titulo}{Práctica 1}
\newcommand{\subtitulo}{El entorno de GNU-Linux}
\newcommand{\materia}{Cómputo paralelo}
\newcommand{\profesor}{}                   % Ej: Dr. Juan Pérez
\newcommand{\fechaEntrega}{17 de febrero del 2026}  % Cambiar por la fecha real

% --- Información institucional (*) ---
\newcommand{\institucion}{Instituto Politécnico Nacional}
\newcommand{\escuela}{Escuela Superior de Cómputo}
\newcommand{\siglasEscuela}{ESCOM}         % Siglas de la escuela
\newcommand{\carrera}{Ingeniería en Inteligencia Artificial}
\newcommand{\grupo}{6BM1}
\newcommand{\lugar}{Ciudad de México}       % Lugar de entrega

% --- Autores (*) ---
% TIP: Agrega o elimina líneas según el número de integrantes.
%      Si es trabajo individual, deja un solo nombre.
%      Separa cada nombre con \\
\newcommand{\listaAutores}{
    Bustillos Cruz Jonatan \\
    Delgado Lucero Cristian Isaac \\
    Frem Cortés José Angel \\
    Luna Gonzales Gabriel Alexis
}

% --- Logos ---
% TIP: Coloca tus logos en imagenes/logos/
%      Si no necesitas algún logo, déjalo vacío {}
\newcommand{\logoIzquierdo}{imagenes/logos/ipn_logo.png}
\newcommand{\logoDerecho}{imagenes/logos/escom_logo.png}

% --- Ruta base de imágenes ---
% TIP: Todas las imágenes se buscarán desde esta carpeta.
%      Puedes usar subcarpetas: \includegraphics{subcarpeta/imagen.png}
\graphicspath{{imagenes/}}
