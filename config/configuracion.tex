% ==========================================
% CONFIGURACIÓN DE LA PLANTILLA
% ==========================================
%
% ADVERTENCIA: Este archivo contiene la configuración base
% de paquetes y estilos. Normalmente NO necesitas editarlo.
%
% Para personalizar tu proyecto, edita: config/proyecto.tex
%
% ==========================================

% ==========================================
% PAQUETES ESENCIALES
% ==========================================

% --- Codificación y tipografía ---
\usepackage[utf8]{inputenc}
\usepackage[T1]{fontenc}

% Aplicar configuración de fuente desde proyecto.tex
% Usa etoolbox para comparar strings
\makeatletter
\newcommand{\@checkusartimes}{}
\let\@checkusartimes\usarTimesNewRoman
\newif\ifusartimes
\def\@truestring{true}
\ifx\@checkusartimes\@truestring
  \usartimestrue
\else
  \usartimesfalse
\fi
\makeatother

\ifusartimes
  % Usar Times New Roman
  \usepackage{newtxtext}       % Times New Roman para texto
  \usepackage{newtxmath}       % Times New Roman para matemáticas
\else
  % Usar fuente por defecto (Latin Modern)
  \usepackage{lmodern}
\fi

% --- Idioma ---
% TIP: Si escribes en otro idioma, cambia 'spanish' aquí.
%      es-tabla cambia "Table" por "Tabla" automáticamente.
\usepackage{csquotes}                       % Requerido por biblatex con babel
\usepackage[spanish,es-tabla]{babel}

% --- Márgenes ---
% TIP: Estándar académico IPN. Ajusta si tu escuela pide otro formato.
\usepackage[top=2.5cm, bottom=2.5cm, left=3cm, right=2.5cm]{geometry}

% --- Gráficos e imágenes ---
\usepackage{graphicx}
\usepackage{float}                          % Para posicionar figuras con [H]

% --- Colores ---
% TIP: xcolor debe cargarse antes de otros paquetes que usen colores.
%      La opción [table] permite colorear filas/columnas en tablas.
\usepackage[table,xcdraw]{xcolor}

% --- Tablas ---
\usepackage{booktabs}                       % Líneas profesionales: \toprule, \midrule, \bottomrule
\usepackage{longtable}                      % Tablas que abarcan varias páginas
\usepackage{multirow}                       % Celdas que abarcan varias filas

% --- Enlaces y referencias ---
\usepackage[hidelinks]{hyperref}            % Enlaces sin bordes de color
\usepackage{url}                            % URLs con salto de línea
\usepackage{seqsplit}                       % Partir cadenas largas sin espacios

% --- Código fuente ---
% TIP: listings es más ligero. minted requiere Python + Pygments instalado.
%      Si no necesitas resaltado avanzado, puedes comentar minted.
\usepackage{listings}
\usepackage{listingsutf8}
\lstset{
  inputencoding=utf8,
  extendedchars=true
}

% Descomenta si tienes Pygments instalado (pip install Pygments)
% y compilas con: pdflatex -shell-escape main.tex
% \usepackage{minted}
% \setminted{
%   breaklines=true,
%   frame=lines,
%   fontsize=\small,
%   tabsize=2
% }

% --- Encabezados y pies de página ---
\usepackage{fancyhdr}

% --- Formato de títulos (para aplicar tamaños personalizados) ---
\usepackage{titlesec}

% Aplicar tamaños de títulos desde proyecto.tex
\titleformat{\chapter}[display]
  {\normalfont\bfseries}{\chaptertitlename\ \thechapter}{20pt}{\fontsize{\tamanoTituloUno}{\dimexpr\tamanoTituloUno*12/10\relax}\selectfont}
\titleformat{\section}
  {\normalfont\bfseries}{\thesection}{1em}{\fontsize{\tamanoTituloDos}{\dimexpr\tamanoTituloDos*12/10\relax}\selectfont}
\titleformat{\subsection}
  {\normalfont\bfseries}{\thesubsection}{1em}{\fontsize{\tamanoTituloTres}{\dimexpr\tamanoTituloTres*12/10\relax}\selectfont}

% --- Listas mejoradas ---
\usepackage{enumitem}

% --- Matemáticas ---
\ifusartimes
  % newtxmath ya carga amsmath e incluye los símbolos de AMS
  % No cargar amsfonts/amssymb para evitar conflictos
\else
  % Cargar paquetes matemáticos para Latin Modern
  \usepackage{amsmath}
  \usepackage{amsfonts}
  \usepackage{amssymb}
\fi

% --- Apéndices ---
\usepackage[titletoc]{appendix}

% --- Bibliografía ---
% TIP: backend=biber es más moderno que bibtex.
%      Compila con: pdflatex → biber → pdflatex → pdflatex
\usepackage[backend=biber, style=ieee, sorting=none]{biblatex}
\addbibresource{referencias.bib}

% ==========================================
% CONFIGURACIÓN DE ESTILOS
% ==========================================

% --- Colores para código ---
\definecolor{codegreen}{rgb}{0,0.6,0}
\definecolor{codegray}{rgb}{0.5,0.5,0.5}
\definecolor{codepurple}{rgb}{0.58,0,0.82}
\definecolor{backcolour}{rgb}{0.95,0.95,0.92}

% --- Estilo de listings ---
\lstdefinestyle{mystyle}{
    backgroundcolor=\color{backcolour},
    commentstyle=\color{codegreen},
    keywordstyle=\color{magenta},
    numberstyle=\tiny\color{codegray},
    stringstyle=\color{codepurple},
    basicstyle=\ttfamily\footnotesize,
    breakatwhitespace=false,
    breaklines=true,
    captionpos=b,
    keepspaces=true,
    numbers=left,
    numbersep=5pt,
    showspaces=false,
    showstringspaces=false,
    showtabs=false,
    tabsize=2
}
\lstset{style=mystyle}

% --- Encabezados ---
% TIP: headheight=14.2pt evita la advertencia de fancyhdr.
\setlength{\headheight}{14.2pt}
\pagestyle{fancy}
\fancyhf{}
\fancyhead[L]{\leftmark}
\fancyhead[R]{\thepage}
\renewcommand{\headrulewidth}{0.5pt}

% ==========================================
% COMANDOS ÚTILES
% ==========================================

% Comillas tipográficas en español
% Uso: \comillas{texto} → ``texto''
\newcommand{\comillas}[1]{``#1''}

% Texto de código en línea con fondo gris
% Uso: \codigo{mi_variable} → mi_variable con fondo
\newcommand{\codigo}[1]{\colorbox{backcolour}{\texttt{#1}}}

% Nota al margen
% Uso: \nota{Recordar revisar esto}
\newcommand{\nota}[1]{\marginpar{\footnotesize\textit{#1}}}
