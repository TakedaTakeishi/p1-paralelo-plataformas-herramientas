% ==========================================
% CAPÍTULO: MARCO TEÓRICO
% ==========================================

\chapter{Marco Teórico}

% ==========================================
% SECCIÓN 1: Máquinas virtuales
% RESPONSABLE: Bustillos Cruz Jonatan
% ==========================================
\section{Máquinas virtuales}

% TODO (Jonatan): Investiga y escribe sobre máquinas virtuales
% Incluye: definición, tipos, ventajas, desventajas, ejemplos


% ==========================================
% SECCIÓN 2: Distribuciones de GNU-Linux
% RESPONSABLE: Bustillos Cruz Jonatan
% ==========================================
\section{Distribuciones de GNU-Linux}

% TODO (Jonatan): Investiga y escribe sobre las distribuciones de GNU-Linux
% Incluye: qué son, principales distribuciones, características


% ==========================================
% SECCIÓN 3: Requisitos de instalación
% RESPONSABLE: Bustillos Cruz Jonatan
% ==========================================
\section{Requisitos de instalación del sistema operativo GNU-Linux}

% TODO (Jonatan): Investiga y escribe sobre los requisitos de instalación
% Incluye: requisitos mínimos y recomendados de hardware


% ==========================================
% SECCIÓN 4: Estadísticas de uso
% RESPONSABLE: Bustillos Cruz Jonatan
% ==========================================
\section{Estadísticas de uso del sistema operativo GNU-Linux en el mundo}

% TODO (Jonatan): Investiga y escribe sobre estadísticas de uso
% Incluye: porcentaje de uso, sectores donde se usa, tendencias


% ==========================================
% SECCIÓN 5: Escritorios GNOME y KDE
% RESPONSABLE: Bustillos Cruz Jonatan
% ==========================================
\section{Escritorios GNOME y KDE}

% TODO (Jonatan): Investiga y escribe sobre GNOME y KDE
% Incluye: diferencias, ventajas, características de cada uno


% ==========================================
% SECCIÓN 6: Entornos CLI y GUI
% RESPONSABLE: Delgado Lucero Cristian Isaac
% ==========================================
\section{Entornos Command Line Interface y Graphical User Interface}

% TODO (Cristian): Investiga y escribe sobre CLI y GUI
% Incluye: definiciones, diferencias, ventajas y desventajas


% ==========================================
% SECCIÓN 7: Terminal de GNU-Linux
% RESPONSABLE: Delgado Lucero Cristian Isaac
% ==========================================
\section{Terminal de GNU-Linux}

% TODO (Cristian): Investiga y escribe sobre la terminal
% Incluye: qué es, para qué sirve, emuladores de terminal


% ==========================================
% SECCIÓN 8: Tipos de usuario en GNU-Linux
% RESPONSABLE: Delgado Lucero Cristian Isaac
% ==========================================
\section{Tipos de usuario en GNU-Linux}

% TODO (Cristian): Investiga y escribe sobre tipos de usuario
% Incluye: root, usuarios normales, permisos, grupos


% ==========================================
% SECCIÓN 9: Rutas relativas y absolutas
% RESPONSABLE: Delgado Lucero Cristian Isaac
% ==========================================
\section{Rutas relativas y absolutas}

% TODO (Cristian): Investiga y escribe sobre rutas
% Incluye: diferencias, ejemplos, cuándo usar cada una


% ==========================================
% SECCIÓN 10: Redireccionamiento
% RESPONSABLE: Delgado Lucero Cristian Isaac
% ==========================================
\section{Redireccionamiento}

% TODO (Cristian): Investiga y escribe sobre redireccionamiento
% Incluye: >, >>, <, |, ejemplos de uso


% ==========================================
% SECCIÓN 11: Clasificación de comandos
% RESPONSABLE: Delgado Lucero Cristian Isaac
% ==========================================
\section{Clasificación de los comandos en GNU-Linux}

% TODO (Cristian): Investiga y escribe sobre la clasificación de comandos
% Incluye: tipos de comandos, internos, externos, alias


% ==========================================
% SECCIÓN 12: Variables de entorno
% RESPONSABLE: Frem Cortés José Angel
% ==========================================
\section{Variables de entorno}

% TODO (José Angel): Investiga y escribe sobre variables de entorno
% Incluye: qué son, principales variables (PATH, HOME, etc.), cómo configurarlas


% ==========================================
% SECCIÓN 13: Comandos de GNU-Linux (Tabla 1)
% RESPONSABLE: Frem Cortés José Angel
% ==========================================
\section{Comandos de GNU-Linux mostrados en la tabla 1}

% TODO (José Angel): Investiga y escribe sobre los comandos de la Tabla 1
% Describe brevemente qué hace cada comando y para qué sirve
% Los comandos son: cal, clear, apt, rm, date, ifconfig, exit, mv, echo, df,
% ps, more, time, du, ps -fea, less, uname, pstree, man, mkdir, w, kill,
% cat, pico, who, trap, fg, nano, bash, pwd, cd, vi, wc, su, ls, apt-get, sudo

