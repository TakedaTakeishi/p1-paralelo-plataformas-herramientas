% ==========================================
% CAPÍTULO DE EJEMPLO
% ==========================================
% Este capítulo muestra los patrones más comunes de LaTeX.
% Úsalo como referencia y elimínalo cuando ya no lo necesites.
%
% TIP: Cada capítulo es un archivo independiente en capitulos/
%      y se incluye desde main.tex con \input{capitulos/nombre}
%
% TIP: Nunca pongas \begin{document} ni \end{document} aquí.
%      Eso sólo va en main.tex.
% ==========================================

\chapter{Capítulo de ejemplo}
\label{cap:ejemplo}

% TIP: \label{} te permite referenciar este capítulo desde cualquier parte:
%      "Como se menciona en el Capítulo \ref{cap:ejemplo}..."

\section{Texto básico}

Este es un párrafo normal. En \LaTeX{}, los párrafos se separan con una línea en blanco.
No uses \texttt{\textbackslash\textbackslash} para saltos de línea en texto normal,
eso es sólo para tablas y ecuaciones.

% TIP: Para texto en negritas usa \textbf{}, para itálicas \textit{},
%      para código en línea \texttt{} o el comando \codigo{} definido en la plantilla.

Este texto tiene \textbf{negritas}, \textit{itálicas} y \codigo{código en línea}.
También puedes usar las \comillas{comillas tipográficas} definidas en la plantilla.


\section{Figuras}
\label{sec:figuras}

% TIP: Coloca tus imágenes en la carpeta imagenes/
%      LaTeX las encuentra automáticamente gracias a \graphicspath en proyecto.tex
%
% TIP: Usa [H] (requiere paquete float) para forzar la posición exacta.
%      Usa [htbp] para dejar que LaTeX elija la mejor posición.
%      Preferir [htbp] produce documentos mejor maquetados.
%
% TIP: SIEMPRE pon \label DESPUÉS de \caption, nunca antes.
%      De lo contrario, las referencias cruzadas apuntarán al lugar incorrecto.
%
% TIP: No dejes líneas en blanco entre \caption y \label.

% --- Ejemplo de figura simple ---
% \begin{figure}[htbp]
%     \centering
%     \includegraphics[width=0.7\textwidth]{nombre_imagen.png}
%     \caption{Descripción clara de la imagen}
%     \label{fig:mi_figura}
% \end{figure}

% Para referenciar: "Como se muestra en la Figura \ref{fig:mi_figura}..."


\section{Tablas}
\label{sec:tablas}

% TIP: Usa booktabs (\toprule, \midrule, \bottomrule) en lugar de \hline.
%      Se ve mucho más profesional.
%
% TIP: No uses líneas verticales (|) en tablas académicas.
%
% TIP: Para tablas largas que cruzan páginas, usa longtable.
%      Para tablas cortas, usa table + tabular.

% --- Ejemplo de tabla corta ---
\begin{table}[htbp]
\centering
\begin{tabular}{@{}llr@{}}
\toprule
\textbf{Elemento} & \textbf{Categoría} & \textbf{Valor} \\ \midrule
Primer elemento    & Tipo A             & 100             \\
Segundo elemento   & Tipo B             & 200             \\
Tercer elemento    & Tipo A             & 150             \\ \midrule
\textbf{Total}     &                    & \textbf{450}    \\ \bottomrule
\end{tabular}
\caption{Ejemplo de tabla con booktabs}
\label{tab:ejemplo}
\end{table}

% --- Ejemplo de tabla larga (multi-página) ---
% \begin{longtable}{@{}p{2cm}p{8cm}p{3cm}@{}}
% \toprule
% \textbf{ID} & \textbf{Descripción} & \textbf{Estado} \\ \midrule
% \endhead  % <-- Esto repite el encabezado en cada página
% Fila 1 & Descripción de la fila 1 & Activo \\
% Fila 2 & Descripción de la fila 2 & Inactivo \\ \bottomrule
% \caption{Ejemplo de tabla larga}
% \label{tab:ejemplo_largo}
% \end{longtable}


\section{Ecuaciones}
\label{sec:ecuaciones}

% TIP: Ecuación en línea: $E = mc^2$
% TIP: Ecuación centrada sin número: \[ E = mc^2 \]
% TIP: Ecuación centrada con número (referenciable): usar equation

La famosa ecuación de Einstein es $E = mc^2$.

% --- Ejemplo de ecuación numerada ---
\begin{equation}
    \sum_{i=1}^{n} i = \frac{n(n+1)}{2}
    \label{eq:suma_gauss}
\end{equation}

La Ecuación \ref{eq:suma_gauss} muestra la suma de Gauss.


\section{Listas}
\label{sec:listas}

% TIP: itemize = viñetas, enumerate = numerada, description = con título

% --- Lista con viñetas ---
\begin{itemize}
    \item Primer punto
    \item Segundo punto
    \item Tercer punto con sublista:
    \begin{itemize}
        \item Subpunto A
        \item Subpunto B
    \end{itemize}
\end{itemize}

% --- Lista numerada ---
\begin{enumerate}
    \item Primer paso
    \item Segundo paso
    \item Tercer paso
\end{enumerate}


\section{Código fuente}
\label{sec:codigo}

% TIP: Usa lstlisting para bloques de código.
%      El estilo 'mystyle' ya está configurado en configuracion.tex.
%
% TIP: Para lenguajes soportados, usa language=Python, language=Java, etc.
%      Esto activa el resaltado de sintaxis automático.
%
% TIP: Si necesitas resaltado más avanzado, activa minted en configuracion.tex
%      (requiere Python + Pygments instalado).

\begin{lstlisting}[language=Python, caption={Ejemplo de código Python}, label={lst:ejemplo_python}]
def fibonacci(n):
    """Calcula el n-esimo numero de Fibonacci."""
    if n <= 1:
        return n
    return fibonacci(n - 1) + fibonacci(n - 2)

# Ejemplo de uso
for i in range(10):
    print(f"F({i}) = {fibonacci(i)}")
\end{lstlisting}

El Listado \ref{lst:ejemplo_python} muestra un ejemplo de código con resaltado de sintaxis.


\section{Citas bibliográficas}
\label{sec:citas}

% TIP: Las referencias se definen en referencias.bib
%      y se citan con \cite{clave} o \textcite{clave}.
%
% TIP: Ejemplo de entrada en referencias.bib:
%
% @book{knuth1997,
%     author    = {Donald E. Knuth},
%     title     = {The Art of Computer Programming},
%     publisher = {Addison-Wesley},
%     year      = {1997},
%     volume    = {1}
% }
%
% Luego en el texto: "Según Knuth \cite{knuth1997}..."
%
% TIP: Para que las citas aparezcan, descomenta \input{common/bibliografia}
%      en main.tex y compila con: pdflatex → biber → pdflatex → pdflatex


\section{Referencias cruzadas}
\label{sec:referencias}

% TIP: Puedes referenciar cualquier elemento con \label y \ref:
%      Capítulos: Capítulo \ref{cap:ejemplo}
%      Secciones: Sección \ref{sec:figuras}
%      Figuras:   Figura \ref{fig:mi_figura}
%      Tablas:    Tabla \ref{tab:ejemplo}
%      Ecuaciones: Ecuación \ref{eq:suma_gauss}
%      Código:    Listado \ref{lst:ejemplo_python}
%
% TIP: Las referencias se actualizan al compilar DOS veces.
%      Si ves "??" en lugar de números, compila de nuevo.

Este capítulo (Capítulo \ref{cap:ejemplo}) contiene ejemplos de:
tablas (Tabla~\ref{tab:ejemplo}),
ecuaciones (Ecuación~\ref{eq:suma_gauss})
y código (Listado~\ref{lst:ejemplo_python}).

% TIP: Usa ~ (tilde) entre "Figura" y \ref para evitar que
%      se separen en líneas diferentes. Ej: Figura~\ref{fig:x}
