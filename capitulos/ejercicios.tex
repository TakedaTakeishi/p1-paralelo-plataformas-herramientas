% ==========================================
% CAPÍTULO: EJERCICIOS PRÁCTICOS
% ==========================================

\chapter{Ejercicios Prácticos}

% ==========================================
% EJERCICIO 15: Rutas absolutas y relativas
% RESPONSABLE: Frem Cortés José Angel
% ==========================================
\section{Rutas absolutas y relativas}

% TODO (José Angel): Escribe y explica:
% - Al menos 3 ejemplos de rutas absolutas
% - Al menos 3 ejemplos de rutas relativas
% Incluye capturas de pantalla mostrando el uso de estas rutas


% ==========================================
% EJERCICIO 16: Borrado con comodines
% RESPONSABLE: Frem Cortés José Angel
% ==========================================
\section{Borrado de archivos con comodines}

% TODO (José Angel): Escribe y explica 3 ejemplos de borrado de archivos
% utilizando el comando rm y los comodines "*" y "?"
% Incluye capturas de pantalla de cada ejemplo


% ==========================================
% EJERCICIO 17: Redireccionamiento
% RESPONSABLE: Frem Cortés José Angel
% ==========================================
\section{Redireccionamiento con \textgreater{} y \textgreater\textgreater}

% TODO (José Angel): Escribe y explica 2 ejemplos de redireccionamiento
% utilizando los comandos > y >>
% Muestra la diferencia entre ambos con capturas de pantalla


% ==========================================
% EJERCICIO 18: Instalación por línea de comandos
% RESPONSABLE: Frem Cortés José Angel
% ==========================================
\section{Instalación de software por línea de comandos}

% TODO (José Angel): Instala algún software en línea de comandos
% (por ejemplo, gimp o konsole)
% Documenta el proceso completo con capturas de pantalla


% ==========================================
% EJERCICIO 19: Instalación por entorno gráfico
% RESPONSABLE: Luna Gonzales Gabriel Alexis
% ==========================================
\section{Instalación de software usando el entorno gráfico}

% TODO (Gabriel): Instala algún software usando el entorno gráfico
% Documenta el proceso completo con capturas de pantalla


% ==========================================
% EJERCICIO 20: Jerarquía de directorios
% RESPONSABLE: Luna Gonzales Gabriel Alexis
% ==========================================
\section{Mover y copiar archivos en la jerarquía de directorios}

% TODO (Gabriel): Escribe al menos 5 ejemplos utilizando la jerarquía
% en los directorios para mover y copiar archivos y directorios
% desde la terminal
% Incluye capturas de pantalla de cada ejemplo


% ==========================================
% EJERCICIO 21: Gestión de usuarios
% RESPONSABLE: Luna Gonzales Gabriel Alexis
% ==========================================
\section{Gestión de usuarios}

% TODO (Gabriel): Escribe la secuencia para:
% 1. Agregar un usuario
% 2. Verificar que existe
% 3. Entrar en sesión con el usuario nuevo
% 4. Borrarlo cuando el usuario nuevo todavía está en sesión
% Documenta cada paso con capturas de pantalla


% ==========================================
% EJERCICIO 22: Hola Mundo en C con nano
% RESPONSABLE: Luna Gonzales Gabriel Alexis
% ==========================================
\section{Hola Mundo en Lenguaje C usando nano}

% TODO (Gabriel): Escribe la secuencia completa para:
% 1. Crear el archivo
% 2. Editar con nano
% 3. Compilar con gcc
% 4. Ejecutar el programa
% Incluye capturas de pantalla de cada paso y el código fuente


% ==========================================
% EJERCICIO 23: Hola Mundo en C con gedit
% RESPONSABLE: Luna Gonzales Gabriel Alexis
% ==========================================
\section{Hola Mundo en Lenguaje C usando gedit}

% TODO (Gabriel): Escribe la secuencia completa para:
% 1. Crear el archivo
% 2. Editar con gedit
% 3. Compilar con gcc
% 4. Ejecutar el programa
% Incluye capturas de pantalla de cada paso y el código fuente

