% ==========================================
% CAPÍTULO: EJERCICIOS PRÁCTICOS
% ==========================================

\chapter{Ejercicios Prácticos}

% ==========================================
% EJERCICIO 15: Rutas absolutas y relativas
% RESPONSABLE: Frem Cortés José Angel
% ==========================================
\section{Rutas absolutas y relativas}

% TODO (José Angel): Escribe y explica:
% - Al menos 3 ejemplos de rutas absolutas
% - Al menos 3 ejemplos de rutas relativas
% Incluye capturas de pantalla mostrando el uso de estas rutas
% El PDF no pide capturas de pantalla, solamente escribir y explicar cada ejemplo.
1.- Rutas Absolutas: Definen la ubicación exacta desde la raíz del sistema (/). No importa dónde se está trabajando, siempre te llevará al mismo lugar.

Algunos ejemplos son:

1) /home/usuario/documentos/tarea.txt: Ruta completa desde la raíz hasta el archivo.

2) /etc/network/interfaces: Acceso directo a una configuración del sistema.

3) /var/log/syslog: Ubicación fija de un archivo de registro del sistema.

2.- Rutas Relativas: Se basan en la ubicación actual (directorio de trabajo). Si se cambia de carpeta, la ruta relativa también debe cambiar para ser válida.

Algunos ejemplos son:

1) documentos/fotos: Busca la carpeta "fotos" dentro de "documentos", asumiendo que ya se está en la carpeta superior.

2) ./script.sh: El . indica el directorio actual; sirve para ejecutar algo donde se está parado.

3) ../imagenes/logo.png: El .. significa "subir un nivel" en las carpetas y luego buscar en "imagenes". 

% ==========================================
% EJERCICIO 16: Borrado con comodines
% RESPONSABLE: Frem Cortés José Angel
% ==========================================
\section{Borrado de archivos con comodines}

% TODO (José Angel): Escribe y explica 3 ejemplos de borrado de archivos
% utilizando el comando rm y los comodines "*" y "?"
% Incluye capturas de pantalla de cada ejemplo
% El PDF no pide capturas de pantalla, solamente escribir y explicar cada ejemplo.
El comando rm (remove) elimina archivos. Los comodines permiten seleccionar varios archivos a la vez según un patrón.

1.- Comodín *: Representa cualquier número de caracteres (incluyendo cero).

Algunos ejemplos son:

1) rm *.jpg: Borra todos los archivos que terminen en ".jpg" en la carpeta actual.

2) rm notas*: Borra cualquier archivo que empiece con "notas", como "notas1.txt", "notas\_viejas.doc", etc.
    
2.- Comodín ?: Representa exactamente un solo carácter cualquiera.

Un ejemplo es:

3) rm tarea?.pdf: Borra archivos como "tarea1.pdf" o "tareaA.pdf", pero no "tarea10.pdf" (porque tiene dos caracteres después de "tarea"). 

% ==========================================
% EJERCICIO 17: Redireccionamiento
% RESPONSABLE: Frem Cortés José Angel
% ==========================================
\section{Redireccionamiento con \textgreater{} y \textgreater\textgreater}

% TODO (José Angel): Escribe y explica 2 ejemplos de redireccionamiento
% utilizando los comandos > y >>
% Muestra la diferencia entre ambos con capturas de pantalla
% El PDF no pide capturas de pantalla, solamente escribir y explicar cada ejemplo.
1.- Operador > (Sobrescribir): Envía la salida a un archivo. Si el archivo ya existe, borra su contenido anterior y escribe lo nuevo.

Ejemplo: ls > lista\_archivos.txt

Explicación: Crea un archivo llamado "lista\_archivos.txt" con el contenido de la carpeta. Si el archivo ya existía, se pierde lo que tenía antes.

2.- Operador \textgreater\textgreater{} (Añadir): Envía la salida al final del archivo sin borrar lo que ya tiene (modo "append").

Ejemplo: date \textgreater\textgreater{} registro.log

Explicación: Agrega la fecha y hora actual al final del archivo "registro.log" cada vez que se ejecuta, conservando el historial previo.

% ==========================================
% EJERCICIO 18: Instalación por línea de comandos
% RESPONSABLE: Frem Cortés José Angel
% ==========================================
\section{Instalación de software por línea de comandos}

% TODO (José Angel): Instala algún software en línea de comandos
% (por ejemplo, gimp o konsole)
% Documenta el proceso completo con capturas de pantalla


% ==========================================
% EJERCICIO 19: Instalación por entorno gráfico
% RESPONSABLE: Luna Gonzales Gabriel Alexis
% ==========================================
\section{Instalación de software usando el entorno gráfico}

% TODO (Gabriel): Instala algún software usando el entorno gráfico
% Documenta el proceso completo con capturas de pantalla


% ==========================================
% EJERCICIO 20: Jerarquía de directorios
% RESPONSABLE: Luna Gonzales Gabriel Alexis
% ==========================================
\section{Mover y copiar archivos en la jerarquía de directorios}

% TODO (Gabriel): Escribe al menos 5 ejemplos utilizando la jerarquía
% en los directorios para mover y copiar archivos y directorios
% desde la terminal
% Incluye capturas de pantalla de cada ejemplo


% ==========================================
% EJERCICIO 21: Gestión de usuarios
% RESPONSABLE: Luna Gonzales Gabriel Alexis
% ==========================================
\section{Gestión de usuarios}

% TODO (Gabriel): Escribe la secuencia para:
% 1. Agregar un usuario
% 2. Verificar que existe
% 3. Entrar en sesión con el usuario nuevo
% 4. Borrarlo cuando el usuario nuevo todavía está en sesión
% Documenta cada paso con capturas de pantalla


% ==========================================
% EJERCICIO 22: Hola Mundo en C con nano
% RESPONSABLE: Luna Gonzales Gabriel Alexis
% ==========================================
\section{Hola Mundo en Lenguaje C usando nano}

% TODO (Gabriel): Escribe la secuencia completa para:
% 1. Crear el archivo
% 2. Editar con nano
% 3. Compilar con gcc
% 4. Ejecutar el programa
% Incluye capturas de pantalla de cada paso y el código fuente


% ==========================================
% EJERCICIO 23: Hola Mundo en C con gedit
% RESPONSABLE: Luna Gonzales Gabriel Alexis
% ==========================================
\section{Hola Mundo en Lenguaje C usando gedit}

% TODO (Gabriel): Escribe la secuencia completa para:
% 1. Crear el archivo
% 2. Editar con gedit
% 3. Compilar con gcc
% 4. Ejecutar el programa
% Incluye capturas de pantalla de cada paso y el código fuente

