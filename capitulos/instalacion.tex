% ==========================================
% CAPÍTULO: INSTALACIÓN DE GNU-LINUX
% ==========================================

\chapter{Instalación de GNU-Linux}

\textit{Nota: Esta fase de la práctica se realiza en equipo, sin embargo, como todos los integrantes realizarán actividades en el sistema operativo, todos deben instalar el sistema operativo y colocar resultados en el archivo del reporte de práctica.}

\vspace{1cm}

% ==========================================
% INSTALACIÓN - Bustillos Cruz Jonatan
% ==========================================
\section{Instalación - Bustillos Cruz Jonatan}

% TODO (Jonatan): Coloca aquí tus capturas de pantalla del proceso de instalación
% Incluye:
% - Tipo de instalación (máquina virtual o partición física)
% - Distribución elegida y versión
% - Capturas de cada paso importante de la instalación
% - Configuración final
% - Actualización del sistema

% Ejemplo de cómo insertar una imagen:
% \begin{figure}[H]
%     \centering
%     \includegraphics[width=0.8\textwidth]{instalacion/jonatan_paso1.png}
%     \caption{Descripción del paso 1}
%     \label{fig:jonatan_paso1}
% \end{figure}


% ==========================================
% INSTALACIÓN - Delgado Lucero Cristian Isaac
% ==========================================
\section{Instalación - Delgado Lucero Cristian Isaac}

% TODO (Cristian): Coloca aquí tus capturas de pantalla del proceso de instalación
% Incluye:
% - Tipo de instalación (máquina virtual o partición física)
% - Distribución elegida y versión
% - Capturas de cada paso importante de la instalación
% - Configuración final
% - Actualización del sistema


% ==========================================
% INSTALACIÓN - Frem Cortés José Angel
% ==========================================
\section{Instalación - Frem Cortés José Angel}

% TODO (José Angel): Coloca aquí tus capturas de pantalla del proceso de instalación
% Incluye:
% - Tipo de instalación (máquina virtual o partición física)
% - Distribución elegida y versión
% - Capturas de cada paso importante de la instalación
% - Configuración final
% - Actualización del sistema
<<<<<<< HEAD
La instalación que realicé fue mediante una partición física del disco duro de mi PC.

Elegí Ubuntu ya que es uno de los softwares más utilizadas para trabajar con Linux y fue el que me recomendaron mis compañeros.

La siguiente imagen muestra que estoy creando una memoria swap para respaldar la memoria RAM de mi PC, ya que no cuento con mucha memoria y se me recomendó realizar esta partición para garantizar el seguir trabajando.
\begin{figure}[H]
    \centering
    \includegraphics[width=0.8\textwidth]{instalacion/frem_paso1.jpg}
    \caption{Partición para swap. Fuente: Elaboración propia.}
    \label{fig:frem_paso1}
\end{figure}

A continuación se puede ver el aviso en la configuración de Linux para revisar las particiones realizadas al momento de su instalación.
\begin{figure}[H]
    \centering
    \includegraphics[width=0.8\textwidth]{instalacion/frem_paso2.jpg}
    \caption{Revisión de las particiones del Linux. Fuente: Elaboración propia.}
    \label{fig:frem_paso2}
\end{figure}

Por último se logró obtener el menú para elegir el sistema operativo con el cual se desea trabajar al momento de prender la computadora. Se puede apreciar que están las opciones para elegir entre Windows y Ubuntu.
\begin{figure}[H]
    \centering
    \includegraphics[width=0.8\textwidth]{instalacion/frem_paso3.jpg}
    \caption{Menú de inicio de la PC. Fuente: Elaboración propia.}
    \label{fig:frem_paso3}
\end{figure}
=======

>>>>>>> 1c7df2a6d6c3c5796c853af86b15b2ee525073b5

% ==========================================
% INSTALACIÓN - Luna Gonzales Gabriel Alexis
% ==========================================
\section{Instalación - Luna Gonzales Gabriel Alexis}

% TODO (Gabriel): Coloca aquí tus capturas de pantalla del proceso de instalación
% Incluye:
% - Tipo de instalación (máquina virtual o partición física)
% - Distribución elegida y versión
% - Capturas de cada paso importante de la instalación
% - Configuración final
% - Actualización del sistema

