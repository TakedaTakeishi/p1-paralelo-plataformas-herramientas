% ==========================================
% CAPÍTULO: CONCLUSIONES
% ==========================================

\chapter{Conclusiones}

% ==========================================
% CONCLUSIÓN - Bustillos Cruz Jonatan
% ==========================================
\section{Bustillos Cruz Jonatan}

% TODO (Jonatan): Escribe tu conclusión sobre la práctica
% Incluye:
% - Qué aprendiste
% - Dificultades encontradas
% - Aplicaciones prácticas
% - Reflexión personal


% ==========================================
% CONCLUSIÓN - Delgado Lucero Cristian Isaac
% ==========================================
\section{Delgado Lucero Cristian Isaac}

% TODO (Cristian): Escribe tu conclusión sobre la práctica
% Incluye:
% - Qué aprendiste
% - Dificultades encontradas
% - Aplicaciones prácticas
% - Reflexión personal

% ==========================================
% CONCLUSIÓN - Frem Cortés José Angel
% ==========================================
\section{Frem Cortés José Angel}

% TODO (José Angel): Escribe tu conclusión sobre la práctica
% Incluye:
% - Qué aprendiste
% - Dificultades encontradas
% - Aplicaciones prácticas
% - Reflexión personal
Mediante esta práctica, logré dominar la instalación del sistema operativo Ubuntu y familiarizarme con el uso de la terminal. Aprendí a ejecutar comandos básicos y a gestionar archivos mediante operadores de redirección y comodines, herramientas fundamentales para la administración eficiente de sistemas Linux.

El principal reto técnico consistió en adaptar la instalación a una computadora Acer de modelo antiguo. Aunque el equipo había sido actualizado previamente, el proceso de particionamiento requirió un análisis detallado. Asimismo, la transición desde el ecosistema de Windows representó una dificultad en términos de memoria muscular, especialmente al adaptarme a los nuevos atajos de teclado y la lógica de la interfaz.

Gracias al apoyo de mi compañero Jonatan, implementamos con éxito una partición de memoria swap. Esta configuración es vital para asegurar la estabilidad del sistema en equipos con recursos limitados, evitando bloqueos durante el uso de aplicaciones escolares. Además, aproveché la amplia disponibilidad de espacio en el disco duro para garantizar una segmentación adecuada que permitiera un desempeño óptimo a largo plazo.

En mi opinión, la experiencia ha sido muy satisfactoria al notar que Ubuntu ofrece una mayor fluidez y rapidez en comparación con Windows, incluso en hardware limitado. Aunque aún me encuentro en proceso de superar la costumbre de los atajos tradicionales, considero que Linux es una herramienta poderosa que ha revitalizado mi computadora, dejándola totalmente funcional para mis necesidades académicas.

% ==========================================
% CONCLUSIÓN - Luna Gonzales Gabriel Alexis
% ==========================================
\section{Luna Gonzales Gabriel Alexis}

% TODO (Gabriel): Escribe tu conclusión sobre la práctica
% Incluye:
% - Qué aprendiste
% - Dificultades encontradas
% - Aplicaciones prácticas
% - Reflexión personal

