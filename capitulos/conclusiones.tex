% ==========================================
% CAPÍTULO: CONCLUSIONES
% ==========================================

\chapter{Conclusiones}

% ==========================================
% CONCLUSIÓN - Bustillos Cruz Jonatan
% ==========================================
\section{Bustillos Cruz Jonatan}

% TODO (Jonatan): Escribe tu conclusión sobre la práctica
% Incluye:
% - Qué aprendiste
% - Dificultades encontradas
% - Aplicaciones prácticas
% - Reflexión personal


% ==========================================
% CONCLUSIÓN - Delgado Lucero Cristian Isaac
% ==========================================
\section{Delgado Lucero Cristian Isaac}

% TODO (Cristian): Escribe tu conclusión sobre la práctica
% Incluye:
% - Qué aprendiste
% - Dificultades encontradas
% - Aplicaciones prácticas
% - Reflexión personal

La realización de esta práctica me pareció particularmente interesante, principalmente porque nunca antes había tenido que instalar un sistema operativo por mi cuenta. Aunque estoy acostumbrado a usar computadoras y software de manera cotidiana, el hecho de preparar una memoria USB booteable, configurar el arranque e instalar GNU-Linux desde cero fue una experiencia nueva y formativa para mí.

El proceso no fue especialmente difícil, aunque sí resultó algo tardado en mi equipo, tanto al momento de bootear la USB como durante la instalación del sistema. A pesar de ello, considero que el tiempo invertido valió completamente la pena, ya que me permitió comprender mejor cómo interactúan el hardware y el sistema operativo durante el proceso de instalación.

Durante la instalación surgieron algunos inconvenientes relacionados con la partición de Windows que ya tenía en mi computadora, lo cual me obligó a revisar y corregir ciertos errores. Curiosamente, este proceso también solucionó algunos problemas que previamente había experimentado con aplicaciones de Microsoft, lo cual fue una consecuencia inesperada pero positiva.

El uso de Ubuntu 24.04 me resultó agradable y, en cierto modo, divertido. Explorar la terminal, ejecutar comandos directamente y comprender el funcionamiento interno del sistema me permitió valorar más la potencia que ofrece GNU-Linux, especialmente en un contexto académico como el de Cómputo Paralelo. Agradezco que el profesor nos haya dejado como tarea instalar el sistema operativo, ya que siento que esta práctica no solo fue técnica, sino también formativa, y definitivamente amplió mi comprensión sobre cómo funciona un entorno de desarrollo real.

En conclusión, esta práctica no solo fortaleció mis conocimientos técnicos, sino que también me dio mayor confianza para trabajar directamente con sistemas operativos, herramientas de línea de comandos y entornos de desarrollo en GNU-Linux.


<<<<<<< HEAD
=======

>>>>>>> 1c7df2a6d6c3c5796c853af86b15b2ee525073b5
% ==========================================
% CONCLUSIÓN - Frem Cortés José Angel
% ==========================================
\section{Frem Cortés José Angel}

% TODO (José Angel): Escribe tu conclusión sobre la práctica
% Incluye:
% - Qué aprendiste
% - Dificultades encontradas
% - Aplicaciones prácticas
% - Reflexión personal
<<<<<<< HEAD
Mediante esta práctica, logré dominar la instalación del sistema operativo Ubuntu y familiarizarme con el uso de la terminal. Aprendí a ejecutar comandos básicos y a gestionar archivos mediante operadores de redirección y comodines, herramientas fundamentales para la administración eficiente de sistemas Linux.

El principal reto técnico consistió en adaptar la instalación a una computadora Acer de modelo antiguo. Aunque el equipo había sido actualizado previamente, el proceso de particionamiento requirió un análisis detallado. Asimismo, la transición desde el ecosistema de Windows representó una dificultad en términos de memoria muscular, especialmente al adaptarme a los nuevos atajos de teclado y la lógica de la interfaz.

Gracias al apoyo de mi compañero Jonatan, implementamos con éxito una partición de memoria swap. Esta configuración es vital para asegurar la estabilidad del sistema en equipos con recursos limitados, evitando bloqueos durante el uso de aplicaciones escolares. Además, aproveché la amplia disponibilidad de espacio en el disco duro para garantizar una segmentación adecuada que permitiera un desempeño óptimo a largo plazo.

En mi opinión, la experiencia ha sido muy satisfactoria al notar que Ubuntu ofrece una mayor fluidez y rapidez en comparación con Windows, incluso en hardware limitado. Aunque aún me encuentro en proceso de superar la costumbre de los atajos tradicionales, considero que Linux es una herramienta poderosa que ha revitalizado mi computadora, dejándola totalmente funcional para mis necesidades académicas.
=======

>>>>>>> 1c7df2a6d6c3c5796c853af86b15b2ee525073b5

% ==========================================
% CONCLUSIÓN - Luna Gonzales Gabriel Alexis
% ==========================================
\section{Luna Gonzales Gabriel Alexis}

% TODO (Gabriel): Escribe tu conclusión sobre la práctica
% Incluye:
% - Qué aprendiste
% - Dificultades encontradas
% - Aplicaciones prácticas
% - Reflexión personal

