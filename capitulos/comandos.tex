% ==========================================
% CAPÍTULO: ENTORNO DE GNU-LINUX - COMANDOS
% ==========================================

\chapter{Entorno de GNU-Linux}

\textit{Nota: A partir de este momento, todo el trabajo es en equipo. Ejecuta el programa terminal y capture la pantalla. Ejecuta en la terminal los comandos de la Tabla 1 explicando su funcionamiento.}

\vspace{1cm}

% ==========================================
% COMANDOS - Bustillos Cruz Jonatan (10 comandos)
% ==========================================
\section{Comandos - Bustillos Cruz Jonatan}

% TODO (Jonatan): Ejecuta y explica los siguientes 10 comandos:
% ps, more, time, du, ps -fea, less, uname, pstree, man, mkdir
%
% Para cada comando:
% 1. Escribe qué hace el comando
% 2. Ejecuta el comando en la terminal
% 3. Captura la pantalla
% 4. Explica el resultado obtenido
%
% Ejemplo:
% \subsection{Comando ps}
% El comando \texttt{ps} muestra los procesos en ejecución...
% \begin{figure}[H]
%     \centering
%     \includegraphics[width=0.8\textwidth]{comandos/jonatan_ps.png}
%     \caption{Ejecución del comando ps}
% \end{figure}
% Explicación del resultado...

\subsection{Comando ps}
% TODO (Jonatan): Completa esta sección

\subsection{Comando more}
% TODO (Jonatan): Completa esta sección

\subsection{Comando time}
% TODO (Jonatan): Completa esta sección

\subsection{Comando du}
% TODO (Jonatan): Completa esta sección

\subsection{Comando ps -fea}
% TODO (Jonatan): Completa esta sección

\subsection{Comando less}
% TODO (Jonatan): Completa esta sección

\subsection{Comando uname}
% TODO (Jonatan): Completa esta sección

\subsection{Comando pstree}
% TODO (Jonatan): Completa esta sección

\subsection{Comando man}
% TODO (Jonatan): Completa esta sección

\subsection{Comando mkdir}
% TODO (Jonatan): Completa esta sección


% ==========================================
% COMANDOS - Delgado Lucero Cristian Isaac (10 comandos)
% ==========================================
\section{Comandos - Delgado Lucero Cristian Isaac}

% TODO (Cristian): Ejecuta y explica los siguientes 10 comandos:
% cal, clear, apt, rm, date, ifconfig, exit, mv, echo, df

\subsection{Comando cal}
% TODO (Cristian): Completa esta sección

El comando \texttt{cal} (calendar) muestra un calendario en formato textual. Por defecto, presenta el mes actual, aunque puede configurarse para mostrar un mes o año específico.

\textbf{Tipo:} Comando externo.

\textbf{Sintaxis general:}
\begin{verbatim}
cal [opciones] [mes] [año]
\end{verbatim}

\textbf{Ejemplo:}
\begin{lstlisting}[language=bash]
cal 2 2026
\end{lstlisting}

Muestra el calendario correspondiente a febrero de 2026.


\subsection{Comando clear}
% TODO (Cristian): Completa esta sección

El comando \texttt{clear} limpia la pantalla de la terminal, desplazando el contenido previo fuera de la vista.

\textbf{Tipo:} Comando externo.

\textbf{Sintaxis:}
\begin{lstlisting}[language=bash]
clear
\end{lstlisting}

Se utiliza principalmente para mantener organizada la visualización en sesiones prolongadas.

\subsection{Comando apt}
% TODO (Cristian): Completa esta sección
El comando \texttt{apt} es una herramienta de gestión de paquetes utilizada en distribuciones basadas en Debian. Permite instalar, actualizar y eliminar software desde repositorios oficiales.

\textbf{Tipo:} Comando externo.

\textbf{Ejemplo:}
\begin{lstlisting}[language=bash]
sudo apt update
sudo apt install gimp
\end{lstlisting}

El primer comando actualiza la lista de paquetes disponibles; el segundo instala el paquete indicado.

\subsection{Comando rm}
% TODO (Cristian): Completa esta sección

El comando \texttt{rm} (remove) elimina archivos o directorios.

\textbf{Tipo:} Comando externo.

\textbf{Sintaxis general:}
\begin{lstlisting}[language=bash]
rm [opciones] archivo
\end{lstlisting}

\textbf{Ejemplo:}
\begin{lstlisting}[language=bash]
rm archivo.txt
rm -r carpeta/
\end{lstlisting}

La opción \texttt{-r} permite eliminar directorios de manera recursiva.

\subsection{Comando date}
% TODO (Cristian): Completa esta sección

El comando \texttt{date} muestra o configura la fecha y hora del sistema.

\textbf{Tipo:} Comando externo.

\textbf{Ejemplo:}
\begin{lstlisting}[language=bash]
date
\end{lstlisting}

Muestra la fecha y hora actuales según la configuración del sistema.

\subsection{Comando ifconfig}
% TODO (Cristian): Completa esta sección

El comando \texttt{ifconfig} (interface configuration) permite visualizar o configurar interfaces de red.

\textbf{Tipo:} Comando externo.

\textbf{Ejemplo:}
\begin{lstlisting}[language=bash]
ifconfig
\end{lstlisting}

Muestra información como dirección IP, máscara de red y estado de las interfaces. En sistemas modernos puede ser reemplazado por el comando \texttt{ip}.

\subsection{Comando exit}
% TODO (Cristian): Completa esta sección

El comando \texttt{exit} finaliza la sesión actual del shell o cierra la terminal.

\textbf{Tipo:} Comando interno del shell.

\textbf{Ejemplo:}
\begin{lstlisting}[language=bash]
exit
\end{lstlisting}

\subsection{Comando mv}
% TODO (Cristian): Completa esta sección

El comando \texttt{mv} (move) permite mover o renombrar archivos y directorios.

\textbf{Tipo:} Comando externo.

\textbf{Ejemplo:}
\begin{lstlisting}[language=bash]
mv archivo.txt carpeta/
mv archivo.txt nuevo_nombre.txt
\end{lstlisting}

El primer ejemplo mueve el archivo a un directorio; el segundo lo renombra.


\subsection{Comando echo}
% TODO (Cristian): Completa esta sección

El comando \texttt{echo} imprime texto en la salida estándar.

\textbf{Tipo:} Comando interno del shell (aunque también puede existir versión externa).

\textbf{Ejemplo:}
\begin{lstlisting}[language=bash]
echo "Hola Mundo"
\end{lstlisting}

Se utiliza frecuentemente para mostrar mensajes o generar contenido mediante redireccionamiento.

\subsection{Comando df}
% TODO (Cristian): Completa esta sección
El comando \texttt{df} (disk free) muestra el espacio disponible y utilizado en los sistemas de archivos montados.

\textbf{Tipo:} Comando externo.

\textbf{Ejemplo:}
\begin{lstlisting}[language=bash]
df -h
\end{lstlisting}

La opción \texttt{-h} presenta los valores en formato legible (MB, GB).


% ==========================================
% COMANDOS - Frem Cortés José Angel (9 comandos)
% ==========================================
\section{Comandos - Frem Cortés José Angel}

% TODO (José Angel): Ejecuta y explica los siguientes 9 comandos:
% w, kill -l -9, cat, pico, who, trap -l, fg, nano, bash

\subsection{Comando w}
% TODO (José Angel): Completa esta sección
<<<<<<< HEAD
Este comando proporciona un resumen detallado del estado del sistema y de los usuarios con sesiones activas.
Un ejemplo de su uso se puede apreciar en la siguiente imagen:
\begin{figure}[H]
    \centering
    \includegraphics[width=0.8\textwidth]{comandos/comando_w.png}
    \caption{Resultado en terminal del comando w. Fuente: Elaboración propia.}
    \label{fig:frem_comando1}
\end{figure}

\subsection{Comando kill -l -9}
% TODO (José Angel): Completa esta sección
El flag -l sirve para listar todas las señales que puedes enviar. Te muestra una lista numerada (como SIGKILL, SIGTERM, etc.).

El -9 envía la señal SIGKILL, que fuerza el cierre de un proceso de forma inmediata. Se usa cuando un programa se trabó y no quiere cerrar por las buenas.

Un ejemplo de su uso se puede apreciar en la siguiente imagen:
\begin{figure}[H]
    \centering
    \includegraphics[width=0.8\textwidth]{comandos/comando_kill_-l_-9.png}
    \caption{Resultado en terminal del comando kill -l -9. Fuente: Elaboración propia.}
    \label{fig:frem_comando2}
\end{figure}

\subsection{Comando cat}
% TODO (José Angel): Completa esta sección
Sirve para ver el contenido de un archivo de texto rápido sin abrir un editor.
Un ejemplo de su uso se puede apreciar en la siguiente imagen:
\begin{figure}[H]
    \centering
    \includegraphics[width=0.8\textwidth]{comandos/comando_cat.png}
    \caption{Resultado en terminal del comando cat. Fuente: Elaboración propia.}
    \label{fig:frem_comando3}
\end{figure}

\subsection{Comando pico}
% TODO (José Angel): Completa esta sección
Es un editor de texto súper sencillo (estilo Bloc de Notas).
Un ejemplo de su uso se puede apreciar en la siguiente imagen (aunque se recomienda mejor ver el resultado del comando nano):
\begin{figure}[H]
    \centering
    \includegraphics[width=0.8\textwidth]{comandos/comando_pico.png}
    \caption{Resultado en terminal del comando pico. Fuente: Elaboración propia.}
    \label{fig:frem_comando4}
\end{figure}

\subsection{Comando who}
% TODO (José Angel): Completa esta sección
Se utiliza para listar exclusivamente a los usuarios que han iniciado sesión en el sistema en tiempo real. Un ejemplo de su uso se puede apreciar en la siguiente imagen:
\begin{figure}[H]
    \centering
    \includegraphics[width=0.8\textwidth]{comandos/comando_who.png}
    \caption{Resultado en terminal del comando who. Fuente: Elaboración propia.}
    \label{fig:frem_comando5}
\end{figure}

\subsection{Comando trap -l}
% TODO (José Angel): Completa esta sección
Su funcionamiento lo vi similar al comando de kill. Un ejemplo de su uso se puede apreciar en la siguiente imagen:
\begin{figure}[H]
    \centering
    \includegraphics[width=0.8\textwidth]{comandos/comando_trap_-l.png}
    \caption{Resultado en terminal del comando trap -l. Fuente: Elaboración propia.}
    \label{fig:frem_comando6}
\end{figure}

\subsection{Comando fg}
% TODO (José Angel): Completa esta sección
Sirve para traer de vuelta un proceso que se haya pausado con Ctrl + Z en primer plano y seguir trabajando con el.
Un ejemplo de su uso se puede apreciar en la siguiente imagen:
\begin{figure}[H]
    \centering
    \includegraphics[width=0.8\textwidth]{comandos/comando_fg.png}
    \caption{Resultado en terminal del comando fg. Fuente: Elaboración propia.}
    \label{fig:frem_comando7}
\end{figure}

\subsection{Comando nano}
% TODO (José Angel): Completa esta sección
Es un editor de texto súper sencillo (estilo Bloc de Notas). Es básicamente lo mismo que pico. nano es la versión moderna y mejorada de pico.

Un ejemplo de su uso se puede apreciar en la siguiente imagen:
\begin{figure}[H]
    \centering
    \includegraphics[width=0.8\textwidth]{comandos/comando_nano.png}
    \caption{Resultado en terminal del comando nano. Fuente: Elaboración propia.}
    \label{fig:frem_comando8}
\end{figure}

Si el archivo existe, se abre y se puede trabajar en el. De lo contrario, se crea como nuevo; por ejemplo, ver la siguiente figura.
\begin{figure}[H]
    \centering
    \includegraphics[width=0.8\textwidth]{comandos/archivo_ejemplo.txt.png}
    \caption{Se crea un archivo de texto por el uso del comando nano. Fuente: Elaboración propia.}
    \label{fig:frem_comando8_comp}
\end{figure}

\subsection{Comando bash}
% TODO (José Angel): Completa esta sección
Es el intérprete de comandos por defecto en casi todo Linux. Aunque realmente no vi que hiciera gran cosa este comando.

Un ejemplo de su uso se puede apreciar en la siguiente imagen:
\begin{figure}[H]
    \centering
    \includegraphics[width=0.8\textwidth]{comandos/comando_bash.png}
    \caption{Resultado en terminal del comando bash. Fuente: Elaboración propia.}
    \label{fig:frem_comando9}
\end{figure}
=======

\subsection{Comando kill -l -9}
% TODO (José Angel): Completa esta sección

\subsection{Comando cat}
% TODO (José Angel): Completa esta sección

\subsection{Comando pico}
% TODO (José Angel): Completa esta sección

\subsection{Comando who}
% TODO (José Angel): Completa esta sección

\subsection{Comando trap -l}
% TODO (José Angel): Completa esta sección

\subsection{Comando fg}
% TODO (José Angel): Completa esta sección

\subsection{Comando nano}
% TODO (José Angel): Completa esta sección

\subsection{Comando bash}
% TODO (José Angel): Completa esta sección

>>>>>>> 1c7df2a6d6c3c5796c853af86b15b2ee525073b5

% ==========================================
% COMANDOS - Luna Gonzales Gabriel Alexis (9 comandos)
% ==========================================
\section{Comandos - Luna Gonzales Gabriel Alexis}

% TODO (Gabriel): Ejecuta y explica los siguientes 9 comandos:
% pwd, cd, vi, wc, su, ls, apt-get, sudo, ls -la

\subsection{Comando pwd}
% TODO (Gabriel): Completa esta sección

\subsection{Comando cd}
% TODO (Gabriel): Completa esta sección

\subsection{Comando vi}
% TODO (Gabriel): Completa esta sección

\subsection{Comando wc}
% TODO (Gabriel): Completa esta sección

\subsection{Comando su}
% TODO (Gabriel): Completa esta sección

\subsection{Comando ls}
% TODO (Gabriel): Completa esta sección

\subsection{Comando apt-get}
% TODO (Gabriel): Completa esta sección

\subsection{Comando sudo}
% TODO (Gabriel): Completa esta sección

\subsection{Comando ls -la}
% TODO (Gabriel): Completa esta sección

