% ==========================================
% CAPÍTULO: ENTORNO DE GNU-LINUX - COMANDOS
% ==========================================

\chapter{Entorno de GNU-Linux}

\textit{Nota: A partir de este momento, todo el trabajo es en equipo. Ejecuta el programa terminal y capture la pantalla. Ejecuta en la terminal los comandos de la Tabla 1 explicando su funcionamiento.}

\vspace{1cm}

% ==========================================
% COMANDOS - Bustillos Cruz Jonatan (10 comandos)
% ==========================================
\section{Comandos - Bustillos Cruz Jonatan}

% TODO (Jonatan): Ejecuta y explica los siguientes 10 comandos:
% ps, more, time, du, ps -fea, less, uname, pstree, man, mkdir
%
% Para cada comando:
% 1. Escribe qué hace el comando
% 2. Ejecuta el comando en la terminal
% 3. Captura la pantalla
% 4. Explica el resultado obtenido
%
% Ejemplo:
% \subsection{Comando ps}
% El comando \texttt{ps} muestra los procesos en ejecución...
% \begin{figure}[H]
%     \centering
%     \includegraphics[width=0.8\textwidth]{comandos/jonatan_ps.png}
%     \caption{Ejecución del comando ps}
% \end{figure}
% Explicación del resultado...

\subsection{Comando ps}
El comando \texttt{ps} (process status) muestra una instantánea de los procesos en ejecución en la sesión actual del terminal. Por defecto lista el PID (identificador del proceso), el terminal asociado, el tiempo de CPU consumido y el nombre del proceso.

\begin{figure}[H]
    \centering
    \includegraphics[width=0.8\textwidth]{joni/ps}
    \caption{Ejecución del comando \texttt{ps}}
    \label{fig:joni_ps}
\end{figure}

En la salida se obervan los procesos activos en la sesión: el intérprete de comandos \texttt{bash} y el propio proceso \texttt{ps} lanzado para obtener la lista.

\subsection{Comando more}
El comando \texttt{more} es un paginador que permite visualizar el contenido de un archivo de texto o la salida de otro comando de forma paginada, es decir, pantalla por pantalla. El usuario avanza presionando la barra espaciadora y puede salir con la tecla \texttt{q}. A diferencia de \texttt{less}, \texttt{more} solo permite desplazamiento hacia adelante.

\begin{figure}[H]
    \centering
    \includegraphics[width=0.8\textwidth]{joni/more}
    \caption{Ejecución del comando \texttt{more}: primera pantalla del archivo}
    \label{fig:joni_more}
\end{figure}

\begin{figure}[H]
    \centering
    \includegraphics[width=0.8\textwidth]{joni/more2}
    \caption{Ejecución del comando \texttt{more}: avance a la siguiente sección del archivo}
    \label{fig:joni_more2}
\end{figure}

En las figuras \ref{fig:joni_more} y \ref{fig:joni_more2} se observa la visualización paginada de un archivo de texto. En la parte inferior de la pantalla aparece el indicador \texttt{--Más--} junto con el porcentaje de contenido ya visualizado, lo que permite conocer cuánto falta por recorrer. Al presionar la barra espaciadora se avanza a la siguiente pantalla hasta llegar al final del archivo. Además se puede notar que no puede manejar correctamente los valores unicode, mostrando caracteres extraños en lugar de acentos o símbolos especiales.

\subsection{Comando time}
El comando \texttt{time} mide el tiempo que tarda en ejecutarse un programa o comando. Reporta tres valores: \textit{real} (tiempo de reloj transcurrido), \textit{user} (tiempo de CPU consumido en modo usuario) e \textit{sys} (tiempo de CPU consumido en modo kernel).

\begin{figure}[H]
    \centering
    \includegraphics[width=0.8\textwidth]{joni/time}
    \caption{Ejecución del comando \texttt{time}}
    \label{fig:joni_time}
\end{figure}

En el resultado se aprecia el tiempo total transcurrido (\textit{real}), el tiempo que el proceso utilizó la CPU en espacio de usuario (\textit{user}) y el tiempo empleado por llamadas al sistema operativo (\textit{sys}).

\subsection{Comando du}
El comando \texttt{du} (disk usage) reporta el espacio en disco utilizado por archivos y directorios. Sin argumentos adicionales muestra el tamaño de cada subdirectorio del directorio actual, expresado en bloques de 512 bytes o kilobytes según la configuración del sistema.

\begin{figure}[H]
    \centering
    \includegraphics[width=0.8\textwidth]{joni/du}
    \caption{Ejecución del comando \texttt{du}}
    \label{fig:joni_du}
\end{figure}

La salida muestra el espacio ocupado por cada directorio y sus subdirectorios, lo que permite identificar qué carpetas consumen más almacenamiento.

\subsection{Comando ps -fea}
El comando \texttt{ps -fea} extiende la salida básica de \texttt{ps} mostrando \textbf{todos} los procesos del sistema (no solo los de la sesión actual). La opción \texttt{-f} activa el formato completo (UID, PID, PPID, prioridad, hora de inicio, terminal y comando), \texttt{-e} selecciona todos los procesos y \texttt{-a} incluye los procesos de otros usuarios.

\begin{figure}[H]
    \centering
    \includegraphics[width=0.8\textwidth]{joni/ps2}
    \caption{Ejecución del comando \texttt{ps -fea}}
    \label{fig:joni_ps2}
\end{figure}

Se puede observar la lista completa de procesos del sistema con información detallada: el usuario propietario, el PID, el PID del proceso padre (PPID), el instante de inicio y la línea de comando completa con la que fue lanzado cada proceso.

\subsection{Comando less}
El comando \texttt{less} es un paginador avanzado para visualizar archivos de texto o salidas de comandos. Permite desplazarse tanto hacia adelante como hacia atrás con las teclas de dirección o las teclas \texttt{Page Up}/\texttt{Page Down}, buscar texto con \texttt{/} y salir con \texttt{q}. Es más versátil que \texttt{more} porque no carga el archivo completo en memoria.

\begin{figure}[H]
    \centering
    \includegraphics[width=0.8\textwidth]{joni/less1}
    \caption{Ejecución del comando \texttt{less}: navegación por un archivo de texto}
    \label{fig:joni_less1}
\end{figure}

\begin{figure}[H]
    \centering
    \includegraphics[width=0.8\textwidth]{joni/less2}
    \caption{Ejecución del comando \texttt{less}: desplazamiento hacia otra sección del archivo}
    \label{fig:joni_less2}
\end{figure}

En las figuras \ref{fig:joni_less1} y \ref{fig:joni_less2} se aprecia cómo \texttt{less} es muy parecido a \texttt{more}, sólo que es más moderno y diferencia de \texttt{more}, no existe un indicador de porcentaje fijo, sino que el nombre del archivo aparece en la barra inferior y el usuario puede desplazarse hacia arriba o hacia abajo en cualquier momento sin restricciones. Aún así, aún falta mejorar el soporte automático para caracteres unicode, ya que sigue mostrando caracteres extraños en lugar de acentos o símbolos especiales.

\subsection{Comando uname}
El comando \texttt{uname} muestra información sobre el kernel y el sistema operativo en ejecución. Con la opción \texttt{-a} despliega toda la información disponible: nombre del kernel, hostname, versión del kernel, fecha de compilación, arquitectura del hardware y tipo de sistema operativo.

\begin{figure}[H]
    \centering
    \includegraphics[width=0.8\textwidth]{joni/uname}
    \caption{Ejecución del comando \texttt{uname}}
    \label{fig:joni_uname}
\end{figure}

La salida revela el nombre del sistema operativo, el nombre del equipo en la red, la versión exacta del kernel de Linux junto con su fecha de compilación y la arquitectura del procesador (\texttt{x86\_64}).

\subsection{Comando pstree}
El comando \texttt{pstree} visualiza los procesos en ejecución en forma de árbol jerárquico, mostrando la relación padre-hijo entre ellos. Esto facilita comprender cómo los procesos se originan unos a partir de otros, partiendo típicamente del proceso \texttt{init} o \texttt{systemd} como raíz. Con el fin de que no se sature la terminal con todo el árbol, en la figura \ref{fig:joni_pstree} se  utilizó el comando \texttt{head -15} con una unión \texttt{|} para mostrar solo los primeros 15 procesos del árbol.

\begin{figure}[H]
    \centering
    \includegraphics[width=0.8\textwidth]{joni/pstree}
    \caption{Ejecución del comando \texttt{pstree}}
    \label{fig:joni_pstree}
\end{figure}

En el árbol se puede apreciar la jerarquía de procesos: \texttt{systemd} como proceso raíz del sistema del que se derivan todos los demás servicios y procesos de usuario, incluyendo el terminal y los comandos ejecutados dentro de él.

\subsection{Comando man}
El comando \texttt{man} (manual) muestra las páginas del manual del sistema para un comando o función específica. Proporciona documentación detallada que incluye la descripción, sinopsis, opciones disponibles, ejemplos y referencias relacionadas. Es la fuente de referencia principal para aprender el uso de cualquier comando en GNU/Linux.

\begin{figure}[H]
    \centering
    \includegraphics[width=0.8\textwidth]{joni/man1}
    \caption{Página de manual: encabezado de la entrada}
    \label{fig:joni_man1}
\end{figure}

\begin{figure}[H]
    \centering
    \includegraphics[width=0.8\textwidth]{joni/man2}
    \caption{Página de manual: descripción de opciones}
    \label{fig:joni_man2}
\end{figure}

Las figura \ref{fig:joni_man2} muestran la página de manual consultada, donde se distingue la sección \textit{NAME} con la descripción breve del comando, la sección \textit{SYNOPSIS} con su sintaxis y la sección \textit{DESCRIPTION} con la explicación detallada de cada opción.

\subsection{Comando mkdir}
El comando \texttt{mkdir} (make directory) crea uno o más directorios nuevos en el sistema de archivos. Con la opción \texttt{-p} crea también todos los directorios intermedios necesarios en la ruta indicada si estos no existen.

\begin{figure}[H]
    \centering
    \includegraphics[width=0.8\textwidth]{joni/mkdir}
    \caption{Ejecución del comando \texttt{mkdir}}
    \label{fig:joni_mkdir}
\end{figure}

En la captura se observa la creación de un nuevo directorio y la verificación con \texttt{ls} de que el directorio fue creado exitosamente en la ubicación especificada.


% ==========================================
% COMANDOS - Delgado Lucero Cristian Isaac (10 comandos)
% ==========================================
\section{Comandos - Delgado Lucero Cristian Isaac}

% TODO (Cristian): Ejecuta y explica los siguientes 10 comandos:
% cal, clear, apt, rm, date, ifconfig, exit, mv, echo, df

\subsection{Comando cal}
% TODO (Cristian): Completa esta sección

El comando \texttt{cal} (calendar) muestra un calendario en formato textual. Por defecto, presenta el mes actual, aunque puede configurarse para mostrar un mes o año específico.

\textbf{Tipo:} Comando externo.

\textbf{Sintaxis general:}
\begin{verbatim}
cal [opciones] [mes] [año]
\end{verbatim}

\textbf{Ejemplo:}
\begin{lstlisting}[language=bash]
cal 2 2026
\end{lstlisting}

Muestra el calendario correspondiente a febrero de 2026.


\subsection{Comando clear}
% TODO (Cristian): Completa esta sección

El comando \texttt{clear} limpia la pantalla de la terminal, desplazando el contenido previo fuera de la vista.

\textbf{Tipo:} Comando externo.

\textbf{Sintaxis:}
\begin{lstlisting}[language=bash]
clear
\end{lstlisting}

Se utiliza principalmente para mantener organizada la visualización en sesiones prolongadas.

\subsection{Comando apt}
% TODO (Cristian): Completa esta sección
El comando \texttt{apt} es una herramienta de gestión de paquetes utilizada en distribuciones basadas en Debian. Permite instalar, actualizar y eliminar software desde repositorios oficiales.

\textbf{Tipo:} Comando externo.

\textbf{Ejemplo:}
\begin{lstlisting}[language=bash]
sudo apt update
sudo apt install gimp
\end{lstlisting}

El primer comando actualiza la lista de paquetes disponibles; el segundo instala el paquete indicado.

\subsection{Comando rm}
% TODO (Cristian): Completa esta sección

El comando \texttt{rm} (remove) elimina archivos o directorios.

\textbf{Tipo:} Comando externo.

\textbf{Sintaxis general:}
\begin{lstlisting}[language=bash]
rm [opciones] archivo
\end{lstlisting}

\textbf{Ejemplo:}
\begin{lstlisting}[language=bash]
rm archivo.txt
rm -r carpeta/
\end{lstlisting}

La opción \texttt{-r} permite eliminar directorios de manera recursiva.

\subsection{Comando date}
% TODO (Cristian): Completa esta sección

El comando \texttt{date} muestra o configura la fecha y hora del sistema.

\textbf{Tipo:} Comando externo.

\textbf{Ejemplo:}
\begin{lstlisting}[language=bash]
date
\end{lstlisting}

Muestra la fecha y hora actuales según la configuración del sistema.

\subsection{Comando ifconfig}
% TODO (Cristian): Completa esta sección

El comando \texttt{ifconfig} (interface configuration) permite visualizar o configurar interfaces de red.

\textbf{Tipo:} Comando externo.

\textbf{Ejemplo:}
\begin{lstlisting}[language=bash]
ifconfig
\end{lstlisting}

Muestra información como dirección IP, máscara de red y estado de las interfaces. En sistemas modernos puede ser reemplazado por el comando \texttt{ip}.

\subsection{Comando exit}
% TODO (Cristian): Completa esta sección

El comando \texttt{exit} finaliza la sesión actual del shell o cierra la terminal.

\textbf{Tipo:} Comando interno del shell.

\textbf{Ejemplo:}
\begin{lstlisting}[language=bash]
exit
\end{lstlisting}

\subsection{Comando mv}
% TODO (Cristian): Completa esta sección

El comando \texttt{mv} (move) permite mover o renombrar archivos y directorios.

\textbf{Tipo:} Comando externo.

\textbf{Ejemplo:}
\begin{lstlisting}[language=bash]
mv archivo.txt carpeta/
mv archivo.txt nuevo_nombre.txt
\end{lstlisting}

El primer ejemplo mueve el archivo a un directorio; el segundo lo renombra.


\subsection{Comando echo}
% TODO (Cristian): Completa esta sección

El comando \texttt{echo} imprime texto en la salida estándar.

\textbf{Tipo:} Comando interno del shell (aunque también puede existir versión externa).

\textbf{Ejemplo:}
\begin{lstlisting}[language=bash]
echo "Hola Mundo"
\end{lstlisting}

Se utiliza frecuentemente para mostrar mensajes o generar contenido mediante redireccionamiento.

\subsection{Comando df}
% TODO (Cristian): Completa esta sección
El comando \texttt{df} (disk free) muestra el espacio disponible y utilizado en los sistemas de archivos montados.

\textbf{Tipo:} Comando externo.

\textbf{Ejemplo:}
\begin{lstlisting}[language=bash]
df -h
\end{lstlisting}

La opción \texttt{-h} presenta los valores en formato legible (MB, GB).


% ==========================================
% COMANDOS - Frem Cortés José Angel (9 comandos)
% ==========================================
\section{Comandos - Frem Cortés José Angel}

% TODO (José Angel): Ejecuta y explica los siguientes 9 comandos:
% w, kill -l -9, cat, pico, who, trap -l, fg, nano, bash

\subsection{Comando w}
% TODO (José Angel): Completa esta sección
Este comando proporciona un resumen detallado del estado del sistema y de los usuarios con sesiones activas.
Un ejemplo de su uso se puede apreciar en la siguiente imagen:
\begin{figure}[H]
    \centering
    \includegraphics[width=0.8\textwidth]{comandos/comando_w.png}
    \caption{Resultado en terminal del comando w. Fuente: Elaboración propia.}
    \label{fig:frem_comando1}
\end{figure}

\subsection{Comando kill -l -9}
% TODO (José Angel): Completa esta sección
El flag -l sirve para listar todas las señales que puedes enviar. Te muestra una lista numerada (como SIGKILL, SIGTERM, etc.).

El -9 envía la señal SIGKILL, que fuerza el cierre de un proceso de forma inmediata. Se usa cuando un programa se trabó y no quiere cerrar por las buenas.

Un ejemplo de su uso se puede apreciar en la siguiente imagen:
\begin{figure}[H]
    \centering
    \includegraphics[width=0.8\textwidth]{comandos/comando_kill_-l_-9.png}
    \caption{Resultado en terminal del comando kill -l -9. Fuente: Elaboración propia.}
    \label{fig:frem_comando2}
\end{figure}

\subsection{Comando cat}
% TODO (José Angel): Completa esta sección
Sirve para ver el contenido de un archivo de texto rápido sin abrir un editor.
Un ejemplo de su uso se puede apreciar en la siguiente imagen:
\begin{figure}[H]
    \centering
    \includegraphics[width=0.8\textwidth]{comandos/comando_cat.png}
    \caption{Resultado en terminal del comando cat. Fuente: Elaboración propia.}
    \label{fig:frem_comando3}
\end{figure}

\subsection{Comando pico}
% TODO (José Angel): Completa esta sección
Es un editor de texto súper sencillo (estilo Bloc de Notas).
Un ejemplo de su uso se puede apreciar en la siguiente imagen (aunque se recomienda mejor ver el resultado del comando nano):
\begin{figure}[H]
    \centering
    \includegraphics[width=0.8\textwidth]{comandos/comando_pico.png}
    \caption{Resultado en terminal del comando pico. Fuente: Elaboración propia.}
    \label{fig:frem_comando4}
\end{figure}

\subsection{Comando who}
% TODO (José Angel): Completa esta sección
Se utiliza para listar exclusivamente a los usuarios que han iniciado sesión en el sistema en tiempo real. Un ejemplo de su uso se puede apreciar en la siguiente imagen:
\begin{figure}[H]
    \centering
    \includegraphics[width=0.8\textwidth]{comandos/comando_who.png}
    \caption{Resultado en terminal del comando who. Fuente: Elaboración propia.}
    \label{fig:frem_comando5}
\end{figure}

\subsection{Comando trap -l}
% TODO (José Angel): Completa esta sección
Su funcionamiento lo vi similar al comando de kill. Un ejemplo de su uso se puede apreciar en la siguiente imagen:
\begin{figure}[H]
    \centering
    \includegraphics[width=0.8\textwidth]{comandos/comando_trap_-l.png}
    \caption{Resultado en terminal del comando trap -l. Fuente: Elaboración propia.}
    \label{fig:frem_comando6}
\end{figure}

\subsection{Comando fg}
% TODO (José Angel): Completa esta sección
Sirve para traer de vuelta un proceso que se haya pausado con Ctrl + Z en primer plano y seguir trabajando con el.
Un ejemplo de su uso se puede apreciar en la siguiente imagen:
\begin{figure}[H]
    \centering
    \includegraphics[width=0.8\textwidth]{comandos/comando_fg.png}
    \caption{Resultado en terminal del comando fg. Fuente: Elaboración propia.}
    \label{fig:frem_comando7}
\end{figure}

\subsection{Comando nano}
% TODO (José Angel): Completa esta sección
Es un editor de texto súper sencillo (estilo Bloc de Notas). Es básicamente lo mismo que pico. nano es la versión moderna y mejorada de pico.

Un ejemplo de su uso se puede apreciar en la siguiente imagen:
\begin{figure}[H]
    \centering
    \includegraphics[width=0.8\textwidth]{comandos/comando_nano.png}
    \caption{Resultado en terminal del comando nano. Fuente: Elaboración propia.}
    \label{fig:frem_comando8}
\end{figure}

Si el archivo existe, se abre y se puede trabajar en el. De lo contrario, se crea como nuevo; por ejemplo, ver la siguiente figura.
\begin{figure}[H]
    \centering
    \includegraphics[width=0.8\textwidth]{comandos/archivo_ejemplo.txt.png}
    \caption{Se crea un archivo de texto por el uso del comando nano. Fuente: Elaboración propia.}
    \label{fig:frem_comando8_comp}
\end{figure}

\subsection{Comando bash}
% TODO (José Angel): Completa esta sección
Es el intérprete de comandos por defecto en casi todo Linux. Aunque realmente no vi que hiciera gran cosa este comando.

Un ejemplo de su uso se puede apreciar en la siguiente imagen:
\begin{figure}[H]
    \centering
    \includegraphics[width=0.8\textwidth]{comandos/comando_bash.png}
    \caption{Resultado en terminal del comando bash. Fuente: Elaboración propia.}
    \label{fig:frem_comando9}
\end{figure}

% ==========================================
% COMANDOS - Luna Gonzales Gabriel Alexis (9 comandos)
% ==========================================
\section{Comandos - Luna Gonzales Gabriel Alexis}

% TODO (Gabriel): Ejecuta y explica los siguientes 9 comandos:
% pwd, cd, vi, wc, su, ls, apt-get, sudo, ls -la

\subsection{Comando pwd}
% TODO (Gabriel): Completa esta sección

\subsection{Comando cd}
% TODO (Gabriel): Completa esta sección

\subsection{Comando vi}
% TODO (Gabriel): Completa esta sección

\subsection{Comando wc}
% TODO (Gabriel): Completa esta sección

\subsection{Comando su}
% TODO (Gabriel): Completa esta sección

\subsection{Comando ls}
% TODO (Gabriel): Completa esta sección

\subsection{Comando apt-get}
% TODO (Gabriel): Completa esta sección

\subsection{Comando sudo}
% TODO (Gabriel): Completa esta sección

\subsection{Comando ls -la}
% TODO (Gabriel): Completa esta sección

